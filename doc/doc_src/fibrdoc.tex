\chapter {Fibrations}

In {tt Kenzo}, the fibration\index{fibrations} theory is applied in the simplicial frame
and limited to the {\em simplicial principal fiber spaces}.

\section {Notion of fibration}

Let $B$ a simplicial set and $\cal G$ a simplicial group. A {\em (right--hand) simplicial fibration} is
defined by a {\bf twisting operator}\index{fibration!twisting operator}, $\tau : B \longrightarrow {\cal G}$,
of degree $-1$.
For any dimension $n$ and for any element $b \in B_n$, the face and degeneracy operators must
satisfy the following relations:
\begin{eqnarray*}
\partial_i(\tau b) & = & \tau(\partial_i b), \qquad i < n-1, \\
\partial_{n-1}(\tau b) & = & [\tau(\partial_n b)]^{-1}.\tau(\partial_{n-1} b), \\
\eta_i(\tau b)  & = & \tau (\eta_i b), \qquad i \leq n-1, \\
e_n             & = & \tau(\eta_n b).
\end{eqnarray*}
where $e_n$ is the neutral element of ${\cal G}_n$.
\vskip 0.30cm
Now, the twisting operator $\tau$ defines a fibration of base space $B$, of fiber space $\cal G$
and of total space $B \times_{\tau} {\cal G}$, whose simplicial components  are:
$$ (B \times_{\tau} {\cal G})_n = (B \times {\cal G})_n = (B_n \times {\cal G}_n), $$
and whose  face operators are defined,
in any dimension $n$ and for any pair $(b,g),\, b \in B_n,\, g \in {\cal G}_n$ by:
\begin{eqnarray*}
\partial_i(b,g) & = & (\partial_i b, \partial_i g), \quad i < n, \\
\partial_n(b,g) & = & (\partial_n b, \tau(b).\partial_n g).
\end{eqnarray*}
As it is well known, there are $4$ natural choices for such a definition. The action of the
group can be from the right or from the left and the critical index can be the last one ($n$) or
the first one ($0$). In {\tt Kenzo}, the implementor has chosen the non--usual one (i.e. action
on the right hand and critical index, the last one), leading to the
Szczarba formulas, the best choice in the case of the loop spaces.

\section {Building a twisting operator}

To build a twisting operator, one uses the function {\tt build-smmr} designed as a convenient tool
to build a simplicial morphism. The source (keyword {\tt :sorc}) is the base space, a {\bf reduced} simplicial
set, the target (keyword {\tt :trgt}) is the fiber space, a simplicial group. The degree is $-1$ and
the morphism (keyword {\tt :sintr}) is the internal lisp function implementing $\tau$. See the example
below, where $\tau$ is a simple twisting operator, applying in particular the only (geometrical) simplex
of degree $2$  of $S^2$ upon the bar object  $[1] \in (K(\Z_2,1))_1$. \par
In {\tt Kenzo}, the type {\tt FIBRATION} is defined. An object of this type must be of type
{\tt SIMPLICIAL-MRPH}, the degree of the corresponding morphism  being  $-1$. To apply the morphism,
one uses the well known macro {\tt ?} or the special macro {\tt tw-a-sintr3}.
\vskip 0.30cm
{\parindent=0mm
{\leftskip=5mm
{\tt ?} {\em twop dmns gmsm}  \hfill {\em [Macro]} \par}
{\leftskip=15mm
Apply the twisting operator {\em twop} to the geometrical simplex {\em gmsm} in dimension
{\em dmns}. Of course, {\tt ?} works as well with a combination of geometric simplices
as in  {\tt (? {\em twop cmbn})}. \par}
{\leftskip=5mm
{\tt tw-a-sintr3} {\em sintr dmns absm bspn} \hfill {\em [Macro]} \par}
{\leftskip=15mm
Apply the internal lisp function {\em sintr} implementing a twisting o\-pe\-ra\-tor to the
abstract simplex {\em absm} in dimension {\em dmns}. The base point {\em bspn} of the
target simplicial set (a simplicial group) must be given, since the function may return the neutral
element in dimension $dmns - 1$. \par}
}
\newpage

\section {Constructing the total space}

The total space\index{fibration!total space} associated to a twisting operator is constructed
by the function {\tt fibration-total}.
\vskip 0.30cm
{\parindent=0mm
{\leftskip=5mm
{\tt fibration-total} {\em fibration} \hfill {\em [Function]} \par}
{\leftskip=15mm
Build the simplicial set, total space of the fibration. This function
extracts from the simplicial morphism {\em fibration} (a previously defined twisting operator),
the base space (a simplicial set)
and the fiber space (a simplicial group) and constructs
the twisted cartesian product of these simplicial sets.
The face function is built using the internal function {\tt fibration-total-face}.
The non--degenerate simplices of the total space are coded exactly as those of
the non--twisted product ($ B \times {\cal G}$) obtained by the lisp call
{\tt (crts-prdc {\em base} {\em fiber})}. This non-twisted cartesian product is saved
into the slot {\tt grmd} (graded module) of the instance object.
If the base space  and the fiber space  are both of type {\tt KAN}, then the filling function is computed
by the internal function {\tt fibration-kfll} and  the function {\tt fibration-total}
returns  an object of type {\tt KAN}.  \par}
}

\subsection* {Examples}

{\footnotesize\begin{verbatim}

(setf s2 (sphere 2))  ==>

[K1 Simplicial-Set]

(setf kz2 (k-z2-1))  ==>

[K6 Abelian-Simplicial-Group]

(setf tw2 (build-smmr
             :sorc s2
             :trgt kz2
             :degr -1
             :sintr #'(lambda (dmns gmsm) (absm 0 1))
             :orgn '(s2-tw-kz2)))   ==>

[K18 Fibration]
\end{verbatim}}
\newpage
{\footnotesize\begin{verbatim}
(inspect tw2)  ==>

SIMPLICIAL-MRPH @ #x4a54c2 = [K18 Fibration]
   0 Class --------> #<STANDARD-CLASS SIMPLICIAL-MRPH>
   1 ORGN ---------> (S2-TW-KZ2), a proper list with 1 element
   2 IDNM ---------> fixnum 18 [#x00000048]
   3 RSLTS --------> simple T vector (15) = #(#() #() #() ...)
   4 ?-CLNM -------> fixnum 0 [#x00000000]
   5 ???-CLNM -----> fixnum 0 [#x00000000]
   6 STRT ---------> The symbol :GNRT
   7 INTR ---------> The symbol NIL
   8 DEGR ---------> fixnum -1 [#xfffffffc]
   9 TRGT ---------> [K6 Abelian-Simplicial-Group]
  10 SORC ---------> [K1 Simplicial-Set]
  11 SINTR --------> #<Interpreted Function (unnamed) @ #x4a545a>
\end{verbatim}}
The following statement show how to apply the twisting operator to a
{\em geometrical} simplex. But if we have an abstract simplex, one must
use the special function {\tt tw-a-sintr3}.
{\footnotesize\begin{verbatim}
(? tw2 2 's2)  ==>

<AbSm - 1>

(tw-a-sintr3 (sintr tw2) 3 (absm 1 's2) nil)  ==>

<AbSm 0 1>

(tw-a-sintr3 (sintr tw2) 3 (absm 7 '*) nil)  ==>

<AbSm 1-0 NIL>

(dotimes (i 3) (print (tw-a-sintr3 (sintr tw2) 3
                                   (absm i 's2) nil)))  ==>

<AbSm - 1>
<AbSm 0 1>
<AbSm 1 1>
\end{verbatim}}
Let us build the total space of the fibration (a simplicial set). We show some basis and compute
some homology groups. The method to compute the homology groups will be explained in a following
chapter.
{\footnotesize\begin{verbatim}
(setf tot-spc2 (fibration-total tw2))  ==>

[K24 Simplicial-Set]

(inspect tot-spc2)  ==>

SIMPLICIAL-SET @ #x4a87f2 = [K24 Simplicial-Set]
   0 Class --------> #<STANDARD-CLASS SIMPLICIAL-SET>
   1 ORGN ---------> <...>, a proper list with 2 elements
   2 IDNM ---------> fixnum 24 [#x00000060]
   3 EFHM ---------> The symbol :--UNBOUND--
   4 GRMD ---------> [K19 Simplicial-Set]
   5 DFFR ---------> [K25 Morphism (degree -1)]
   6 BSGN ---------> <CrPr - * - 0>, a dotted list with 3 elements
   7 BASIS --------> #<Closure (FLET CRTS-PRDC-BASIS RSLT) @ #x4a735a>
   8 CMPR ---------> #<Closure (FLET CRTS-PRDC-CMPR RSLT) [#'SPHERE-CMPR] @ #x4a732a>
   9 CPRD ---------> [K28 Morphism (degree 0)]
  10 FACE ---------> #<Closure (FLET FIBRATION-TOTAL-FACE RSLT) @ #x4a8752>

(basis tot-spc2 0)  ==>

(<CrPr - * - 0>)

(basis tot-spc2 1)  ==>

(<CrPr 0 * - 1>)

(basis tot-spc2 2)  ==>

(<CrPr - S2 - 2> <CrPr - S2 0 1> <CrPr - S2 1 1> <CrPr - S2 1-0 0> <CrPr 1-0 * - 2>)

(basis tot-spc2 3)  ==>

(<CrPr 0 S2 - 3> <CrPr 0 S2 1 2> <CrPr 0 S2 2 2> <CrPr 0 S2 2-1 1> <CrPr 1 S2 - 3>
 <CrPr 1 S2 0 2> <CrPr 1 S2 2 2> <CrPr 1 S2 2-0 1> <CrPr 2 S2 - 3> <CrPr 2 S2 0 2>
 <CrPr 2 S2 1 2> <CrPr 2 S2 1-0 1> <CrPr 2-1-0 * - 3>)

(homology tot-spc2 0 8) ==>

Homology in dimension 0 :

Component Z

Homology in dimension 1 :

---done---

Homology in dimension 2 :

Component Z

Homology in dimension 3 :

Component Z/4Z

Homology in dimension 4 :

---done---

Homology in dimension 5 :

Component Z/4Z

Homology in dimension 6 :

---done---

Homology in dimension 7 :

Component Z/4Z
\end{verbatim}}
In the following example, we take $K(\Z,1)$ and build a similar twisting o\-pe\-ra\-tor. Note
that in $(K(\Z,1))_1$, the bar object $[1]$ is represented as {\tt (1)} whereas in $K(\Z_2,1)$ it
is {\tt 1}. In this case, it is known that the total space is a model of the sphere $S^3$
(Hopf fibration - see below).
{\footnotesize\begin{verbatim}
(setf s2 (sphere 2))  ==>

[K1 Simplicial-Set]

(setf kz1 (k-z-1))  ==>

[K6 Abelian-Simplicial-Group]

(setf tw1 (build-smmr
             :sorc s2
             :trgt kz1
             :degr -1
             :sintr #'(lambda (dmns gmsm) (absm 0 (list 1)))
             :orgn '(s2-tw-kz1)))  ==>

[K18 Fibration]

(setf tot-spc1 (fibration-total tw1))  ==>

[K24 Simplicial-Set]

(homology tot-spc1 0 6)  ==>

Homology in dimension 0 :

Component Z

Homology in dimension 1 :

---done---

Homology in dimension 2 :

---done---

Homology in dimension 3 :

Component Z

Homology in dimension 4 :

---done---

Homology in dimension 5 :

---done---
\end{verbatim}}
We may also obtain the boundary and faces of elements of the total space:
{\footnotesize\begin{verbatim}
(setf elem (crpr 0 's2 0 '(1 2)))  ==>

<CrPr - S2 - (1 2)>

(? tot-spc1 2 elem)  ==>

----------------------------------------------------------------------{CMBN 1}
<2 * <CrPr 0 * - (2)>>
<-1 * <CrPr 0 * - (3)>>
------------------------------------------------------------------------------

(? tot-spc1 *)  ==>

----------------------------------------------------------------------{CMBN 0}
------------------------------------------------------------------------------

(dotimes (i 3) (print (face tot-spc1 i 2 elem)))  ==>

<AbSm - <CrPr 0 * - (2)>>
<AbSm - <CrPr 0 * - (3)>>
<AbSm - <CrPr 0 * - (2)>>
\end{verbatim}}
\newpage
The system {\tt Kenzo} provides the function {\tt hopf} to build
Hopf fibrations\index{fibrations! of Hopf}. The lisp definition is the following:
{\footnotesize\begin{verbatim}
(defun HOPF (n)
  (declare (fixnum n))
  (the simplicial-mrph
     (build-smmr
        :sorc (sphere 2)
        :trgt (k-z-1)
        :degr -1
        :sintr (hopf-sintr n)
        :orgn `(hopf ,n))))

(defun HOPF-SINTR (n)
  (declare (fixnum n))
  (flet ((rslt (dmns gmsm)
         (declare (ignore dmns gmsm))
         (if (zerop n)
             (absm 1 +empty-list+)
             (absm 0 (list n)))))
         (the sintr #'rslt)))
\end{verbatim}}
In the definition of the lisp function for the slot {\tt sintr}, we see that the simplicial morphism
applies in particular the only geometric simplex in dimension $2$ of $S^2$ onto the bar object
$[n] \in (K(\Z,1))_1$. If $n=0$, this is $\eta_0 [\ ]$, i.e. the $0$--degeneracy of the base point
{\tt NIL}. We give some examples of some fibrations and of the corresponding total spaces in function of the
parameter {\tt n}. First we generate $4$ fibrations and apply the twisting operator
on the geometrical simplex {\tt s2}.
{\footnotesize\begin{verbatim}
(setf h1 (hopf 1) h2 (hopf 2) h3 (hopf 3) h10 (hopf 10))

(mapcar #'(lambda (h) (funcall (sintr (eval h)) 2 's2)) '(h1 h2 h3 h10)) ==>

(<AbSm - (1)> <AbSm - (2)> <AbSm - (3)> <AbSm - (10)>)
\end{verbatim}}
Then we generate the corresponding total space and get some homology groups.
{\footnotesize\begin{verbatim}
(setf t1 (fibration-total h1) t2  (fibration-total h2)
      t3 (fibration-total h3) t10 (fibration-total h10))

(homology t1 0 10)  ==>

Homology in dimension 0 :

Component Z

Homology in dimension 1 :

---done---

Homology in dimension 2 :

---done---

Homology in dimension 3 :

Component Z

Homology in dimension 4 :

---done---

......... all next groups null .....

(homology t2 0 10)  ==>

Homology in dimension 0 :

Component Z

Homology in dimension 1 :

Component Z/2Z

Homology in dimension 2 :

---done---

Homology in dimension 3 :

Component Z

---done---

......... all next groups null .....

(homology t3 0 4)  ==>

Homology in dimension 0 :

Component Z

Homology in dimension 1 :

Component Z/3Z

Homology in dimension 2 :

---done---

Homology in dimension 3 :

Component Z

(homology t10 1)  ==>

Homology in dimension 1 :

Component Z/10Z
\end{verbatim}}

\subsection* {Lisp file concerned in this chapter}

{\tt fibrations.lisp}.
