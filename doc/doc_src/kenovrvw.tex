\setcounter{chapter}{-1}
\chapter {Overview}

In this overview we try to  show, without entering in detailed
explanations, various possibilities of {\tt Kenzo} in algebraic topology.
The lines ended by the symbol $==>$ are the commands typed by the user
(don't type this symbol!).
They are followed by the answer of the program. We have suppressed some
extra informations, in particular the ones printed during the computation
of homology groups.
\vskip 0.5cm
Let us begin by the space $Moore(\Z/2\Z, 3)$ described as a simplicial set
having only three non--degenerate simplices, namely in dimension
$0$, $3$ and $4$. In the  representation created by the software, the $0$--simplex (base point),
the $3$--simplex and the $4$--simplex are respectively labelled
``{\tt *}'', {\tt M3} and {\tt N4}. Two faces of the $4$-simplex {\tt N4} are identified with the
$3$-simplex {\tt M3}, the others being contracted on the base point. To create the simplicial set
one types simply:
{\footnotesize\begin{verbatim}
(setf m23 (moore 2 3))  ==>
\end{verbatim}}
The system answers that a {\tt Kenzo} object has been created, with number $1$ and
type {\tt SIMPLICIAL SET}. This object may be referenced  by the symbol {\tt m23}.
{\footnotesize\begin{verbatim}
[K1 Simplicial-Set]
\end{verbatim}}
We may compute the homology groups of this space, using the underlying chain complex
induced by the simplicial set description. Here we compute the $ H_i$ from $0$ to $4$ included.
When in the answer the component part is void, it means that the corresponding homology group is null.
{\footnotesize\begin{verbatim}
(homology m23 0 5)  ==>

Homology in dimension 0 :

Component Z

--done--

Homology in dimension 1 :

--done--

Homology in dimension 2 :

--done--

Homology in dimension 3 :

Component Z/2Z

--done--

Homology in dimension 4 :

---done---
\end{verbatim}}
As {\tt m23} is a simplicial set, it is possible to create the cartesian product
$m23 \times m23$ by the function {\tt crts-prdc}. This is a new simplicial set.
{\footnotesize\begin{verbatim}
(setf m23xm23 (crts-prdc m23 m23))  ==>

[K10 Simplicial-Set]
\end{verbatim}}
Being a simplicial set, {\tt m23xm23} is also a chain complex object and we may for instance ask for
the basis in dimension $6$
{\footnotesize\begin{verbatim}
(basis m23xm23 6)  ==>

(<CrPr 1-0 N4 3-2 N4> <CrPr 1-0 N4 4-2 N4> <CrPr 1-0 N4 4-3 N4>
 <CrPr 1-0 N4 4-3-2 M3>  <CrPr 1-0 N4 5-2 N4> <CrPr 1-0 N4 5-3 N4>
 <CrPr 1-0 N4 5-3-2 M3> <CrPr 1-0 N4 5-4 N4>
                            ....  ....
 <CrPr 4-1 N4 5-3-2 M3> <CrPr 4-1-0 M3 3-2 N4> <CrPr 4-1-0 M3 5-2 N4>
 <CrPr 4-1-0 M3 5-3 N4> <CrPr 4-1-0 M3 5-3-2 M3> ... ... ...)

(length *)  ==>

230
\end{verbatim}}
As shown by the last command (* means the previous result),
the number of elements of the basis is quite large (230). The user
will note that the basis elements are formed by cartesian products of {\em degenerated simplices}. In
the list, an element like {\tt <CrPr 1-0 N4 5-3-2 M3>} means $\eta_1\eta_0 N4 \times \eta_5\eta_3\eta_2 M3$.
\par
We may construct also the tensor product $m23 \otimes m23$ from the underlying chain complex
of the simplicial set {\tt m23}. This tensor product is a new chain complex and we see that the
basis in dimension $6$ has only one element:
{\footnotesize\begin{verbatim}
(setf t2m23 (tnsr-prdc m23 m23)) ==>

[K3 Chain-Complex]

(basis t2m23 6)  ==>

(<TnPr M3 M3>)
\end{verbatim}}
The Eilenberg-Zilber theorem is used to compute the homology groups of the cartesian
product space: as chain complexes, {\tt m23xm23} and {\tt t2m23} have the same homology groups,
but the computations using the tensor product are considerably faster.
{\footnotesize\begin{verbatim}
(homology m23xm23 0 8)  ==>

Homology in dimension 0 :

Component Z

Homology in dimension 1 :    (meaning: Group null)

Homology in dimension 2 :

Homology in dimension 3 :

Component Z/2Z

Component Z/2Z

Homology in dimension 4 :

Homology in dimension 5 :

Homology in dimension 6 :

Component Z/2Z

Homology in dimension 7 :

Component Z/2Z

---done---
\end{verbatim}}
\newpage
Let us consider now the space $K(\Z, 1)$. This is an Abelian simplicial group created  in {\tt Kenzo}
by the function {\tt k-z}.
In this simplicial group, a  simplex  in dimension $n$
is mathematically represented by a sequence of integers, known as a {\em bar} object:
$$ [a_1 \mid a_2 \mid \ldots \mid a_n]. $$
In {\tt Kenzo}, a non-degenerate simplex of $K(\Z,1)$ in dimension $n$ will be simply
a list of $n$ non-null integers, for instance: {\tt (2 3 4 5)}. In dimension $0$, the only
simplex is {\tt NIL} (the base point).
{\footnotesize\begin{verbatim}
(setf kz1 (k-z 1))  ==>

[K38 Abelian-Simplicial-Group]
\end{verbatim}}
But this object is also a {\em coalgebra} and an {\em algebra}, and we may
see the effect of the respective induced {\em coproduct} and {\em product}:
{\footnotesize\begin{verbatim}
(cprd kz1 4 '(2 3 4 5))  ==>

----------------------------------------------------------------------{CMBN 4}
<1 * <TnPr NIL (2 3 4 5)>>
<1 * <TnPr (2) (3 4 5)>>
<1 * <TnPr (2 3) (4 5)>>
<1 * <TnPr (2 3 4) (5)>>
<1 * <TnPr (2 3 4 5) NIL>>
------------------------------------------------------------------------------

(aprd kz1 6 (tnpr 2 '(1 2) 4 '(3 4 5 6)))  ==>

----------------------------------------------------------------------{CMBN 6}
<1 * (1 2 3 4 5 6)>
<-1 * (1 3 2 4 5 6)>
<1 * (1 3 4 2 5 6)>
<-1 * (1 3 4 5 2 6)>
<1 * (1 3 4 5 6 2)>
<1 * (3 1 2 4 5 6)>
<-1 * (3 1 4 2 5 6)>
<1 * (3 1 4 5 2 6)>
<-1 * (3 1 4 5 6 2)>
<1 * (3 4 1 2 5 6)>
<-1 * (3 4 1 5 2 6)>
<1 * (3 4 1 5 6 2)>
<1 * (3 4 5 1 2 6)>
<-1 * (3 4 5 1 6 2)>
<1 * (3 4 5 6 1 2)>
------------------------------------------------------------------------------
\end{verbatim}}
The printed results are the printed representation of {\em combinations}, i.e.
integer linear combinations of generators resulting from the application of the
morphisms. The degree of the combination is indicated by the information: {\tt CMBN $n$}.
\par
On the same way, we may create the Abelian simplicial groups $K(\Z/2\Z, n)$:
{\footnotesize\begin{verbatim}
(setf k-z2-2 (k-z2 2))  ==>

[K488 Abelian-Simplicial-Group]

(homology k-z2-2 4)  ==>

Homology in dimension 4 :

Component Z/4Z

---done---
\end{verbatim}}

\begin{center}
----------
\end{center}
Let us play now with the sphere $S^3$ and its loop spaces. $S^3$ and $\Omega^2 S^3$ are created by
respective calls to the functions {\tt sphere} and {\tt loop-space}. Then we compute the $H_4$ and $H_5$
of $\Omega^2 S^3$:
{\footnotesize\begin{verbatim}
(setf s3 (sphere 3))  ==>

[K52 Simplicial-Set]

(setf o2s3 (loop-space s3 2))  ==>

[K69 Simplicial-Group]

(homology o2s3 4 6)  ==>

Homology in dimension 4 :

Component Z/3Z

Component Z/2Z

Homology in dimension 5 :

Component Z/3Z

Component Z/2Z

---done---
\end{verbatim}}
\newpage
Let us take now the first loop space $\Omega^1 S^3$
{\footnotesize\begin{verbatim}
(setf s3 (sphere 3))  ==>

[K94 Simplicial-Set]

(setf os3 (loop-space s3))  ==>

[K99 Simplicial-Group]
\end{verbatim}}
In the following instruction, we locate in the symbol {\tt L1} the canonical generator of
$\pi_2 (\Omega^1S^3)$, that is the $2$--simplex coming from the original sphere. In fact, the object
created by the command {\tt (loop3 0 's3 1)} is the ``word'' $S3^1$ belonging to the
Kan simplicial version $G(S^3)$ (a simplicial group) of the loop space $\Omega S^3$.
{\footnotesize\begin{verbatim}
(setf L1  (loop3 0 's3 1))  ==>

<AbSm - <<Loop[S3]>>>
\end{verbatim}}
Let us consider also the $2$--degeneracy of the base point of the loop space. In the printed
result, the user will recognize the degeneracy $\eta_1\eta_0$ of the null loop, base point of
$\Omega^1 S^3$:
{\footnotesize\begin{verbatim}
(setf null-simp (absm 3 +null-loop+))  ==>

<AbSm 1-0 <<Loop>>>
\end{verbatim}}
We may  build now a new space by pasting a disk $D3$ as indicated by the
following call. It means that we ``paste'' to the space {\tt os3} a $3$--simplex named {\tt D3},
the attaching map being described by the list of its faces in dimension $2$.

{\footnotesize\begin{verbatim}
(setf dos3 (disk-pasting os3 3 '<D3>
              (list L1 null-simp L1 null-simp)))  ==>

[K212 Simplicial-Set]
\end{verbatim}}
Let us compute a few homology groups of the new space {\tt dos3}:
{\footnotesize\begin{verbatim}
(homology dos3 2 4)  ==>

Homology in dimension 2 :

Component Z/2Z

Homology in dimension 3 :

---done---
\end{verbatim}}
But more interesting, let us  build the  loop space of the object {\tt dos3} and
let us compute  the ho\-mo\-lo\-gy in dimension $5$:
{\footnotesize\begin{verbatim}
(setf odos3 (loop-space dos3))  ==>

[K230 Simplicial-Group]

(homology odos3 5)  ==>

Homology in dimension 5 :

Component Z/2Z

Component Z/2Z

Component Z/2Z

Component Z/2Z

Component Z/2Z

Component Z/2Z

Component Z

--done--
\end{verbatim}}
\begin{center}
--------
\end{center}
Let us continue with the Kan theory. First, we check that $S^3$ is not of type Kan and that $\Omega S^3$
is indeed of type Kan and a non--Abelian simplicial group.
{\footnotesize\begin{verbatim}
(typep s3 'kan)  ==>

NIL

(typep os3 'kan)  ==>

T

(typep os3 'simplicial-group)  ==>

T

(typep os3 'ab-simplicial-group)  ==>

NIL


\end{verbatim}}
Let us create the word $L2=(S3)^2$, i.e an object belonging to $\Omega S^3$ and let us
apply the product of the underlying algebra upon $L2 \otimes L2$:
{\footnotesize\begin{verbatim}
(setf L2 (loop3 0 's3 2))  ==>

<<Loop[S3\2]>>

(setf square (aprd os3 4 (tnpr 2 L2 2 L2)))  ==>

----------------------------------------------------------------------{CMBN 4}
<1 * <<Loop[1-0 S3\2][3-2 S3\2]>>>
<-1 * <<Loop[2-0 S3\2][3-1 S3\2]>>>       <-------
<1 * <<Loop[2-1 S3\2][3-0 S3\2]>>>
<1 * <<Loop[3-0 S3\2][2-1 S3\2]>>>
<-1 * <<Loop[3-1 S3\2][2-0 S3\2]>>>
<1 * <<Loop[3-2 S3\2][1-0 S3\2]>>>
------------------------------------------------------------------------------
\end{verbatim}}
We see that the result is a linear combination of words composed from degeneracies
of {\tt L2}. The following instruction selects the generator part of the
second element of the previous combination (noted by the reverse arrow).
{\footnotesize\begin{verbatim}
(setf L4 (gnrt (second (cmbn-list square))))  ==>

<<Loop[2-0 S3\2][3-1 S3\2]>>
\end{verbatim}}
Let us use the lisp function {\tt mapcar} (one among the various iteration functions of Lisp) to
create the list of the faces 1, 2, 3 and 4 of the object {\tt L4}, this list is a ``Kan hat''.
{\footnotesize\begin{verbatim}
(setf hat (mapcar #'(lambda (i) (face os3 i 4 l4)) '(1 2 3 4)))  ==>

(<AbSm - <<Loop[1 S3\2][2 S3\2]>>> <AbSm - <<Loop[0 S3\2][2 S3\2]>>>
 <AbSm - <<Loop[0 S3\2][1 S3\2]>>> <AbSm 1 <<Loop[S3\2]>>>)
\end{verbatim}}
The function {\tt kfll} tries to find a filling of this ``Kan hat'', and we see that
the face $2$ of the resulting simplex (which is very different from {\tt L4}) is  the same
as the face $2$ of {\tt L4}.
{\footnotesize\begin{verbatim}
(setf kan-simplex (kfll os3 0 4 hat))  ==>

<AbSm - <<Loop[3-1 S3\2][2-1 S3\-2][2-0 S3\2][1-0 S3\-2][2-1 S3\2][3-1 S3\-2]
              [1-0 S3\2][3-1 S3\2][3-0 S3\-2][1-0 S3\-2][3-0 S3\2][2-0 S3\-2]
              [1-0 S3\2][3-0 S3\-2][2-0 S3\2][3-0 S3\2]>>>

(face os3 2 4 kan-simplex)  ==>

<AbSm - <<Loop[0 S3\2][2 S3\2]>>>

(second hat)  ==>

<AbSm - <<Loop[0 S3\2][2 S3\2]>>>
\end{verbatim}}
\begin{center}
--------
\end{center}
Let ${\cal G}$ be a simplicial group $0$--reduced. $\Omega S^3$ is such a group.
The software  {\tt Kenzo} allows
the construction of the universal bundle $\bar{\cal W}{\cal G}$, i.e. the
classifying space of ${\cal G}$. In our case, as $\Omega S^3$ in non--Abelian, the
result is not a simplicial group but only a simplicial set. We verify that the $H_4$ is null.
{\footnotesize\begin{verbatim}
(setf cls-os3 (classifying-space os3))  ==>

[K598 Simplicial-Set]

(typep cls-os3 'simplicial-group)  ==>

NIL

(homology cls-os3 4)  ==>

Homology in dimension 4 :

---done---
\end{verbatim}}
\begin{center}
--------
\end{center}
\vskip 0.40cm
Let us end this short overview with an example of computation of homotopy groups.
The method used in {\tt Kenzo} is the Whitehead tower.
In this current version only the case where the first non--null
homology group (in non--null dimension) is $\Z$ or $\Z/{2 \Z}$ can be processed; however if this
homology group is a direct sum of several copies of $\Z$ or $\Z/{2 \Z}$, then the corresponding
stage of the Whitehead tower may also be constructed step by step.
\vskip 0.40cm
We take again $Moore(\Z/2\Z, 3)$ whose $H_3$ is $\Z/2\Z$. First the fundamental cohomology
class is constructed:
{\footnotesize\begin{verbatim}
(setf ch3 (chml-clss m23 3))  ==>

[K729 Cohomology-Class on K1 of degree 3]
\end{verbatim}}
Then the function {\tt z2-whitehead} is called to build a fibration over the simplicial set {\tt m23}
canonically associated to the cohomology class {\tt ch3}.
{\footnotesize\begin{verbatim}
(setf f3 (z2-whitehead m23 ch3))  ==>

[K730 Fibration K1 -> K488]
\end{verbatim}}
Then the total space of the fibration is built:
{\footnotesize\begin{verbatim}
(setf x4 (fibration-total f3))  ==>

[K736 Simplicial-Set]
\end{verbatim}}
The $H_4$ of this total space is the $\pi_4$ of $Moore(\Z/2\Z)$:
{\footnotesize\begin{verbatim}
(homology x4 3 5)  ==>

Homology in dimension 3 :

---done---

Homology in dimension 4 :

Component Z/2Z

---done---
\end{verbatim}}
We may now iterate the process, to compute the $\pi_5$ of $Moore(\Z/2\Z)$:
{\footnotesize\begin{verbatim}
(setf ch4 (chml-clss x4 4))  ==>

[K817 Cohomology-Class on K802 of degree 4]

(setf f4 (z2-whitehead x4 ch4))  ==>

[K832 Fibration K736 -> K818]

(setf x5 (fibration-total f4))  ==>

[K838 Simplicial-Set]

(homology x5 4 6)  ==>

Homology in dimension 4 :

---done---

Homology in dimension 5 :

Component Z/4Z

---done---
\end{verbatim}}
So $\pi_5(Moore(\Z/2\Z))$ is $\Z/4\Z$.
