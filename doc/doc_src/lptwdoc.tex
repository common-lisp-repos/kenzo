\chapter {Loop spaces fibrations}

\section{The canonical loop space fibration}

Let $X$ be a  simplicial set. The software {\tt Kenzo} implements the
{\bf canonical} twisted cartesian product\index{twisted cartesian product} of $X$ by its loop space $GX$,
denoted $X \times_\tau GX$, giving the Kan\footnote {{\bf Daniel M. Kan}. {\em A combinatorial definition of
homotopy groups}, Ann. of Math., 1958, vol. 67, pp 282--312.} model of the contractible
path space of $X$.
The {\em canonical fibration}
$$ GX\quad \hookrightarrow\quad X \times_\tau GX  \hskip 1.2em \longrightarrow \hskip -1.2em \longrightarrow
   \quad X,$$
is defined by the trivial application $\tau$, where $\tau(x)$ is the  {\em letter} $x$ in $GX$.
This twisted cartesian product has the same simplices as the non-twisted one.
So both associated chain complexes
are equal as graded modules. The essential difference  resides in the differential morphism.
More precisely, the face operator $\partial_n$ behaves differently in the twisted cartesian product.
Let $(x,g) \in X \times_\tau GX$, the rule for the face operators are:
\begin{eqnarray*}
\partial_i(x,g) & =  & (\partial_i x, \partial_i g), \qquad i<n, \\
\partial_n(x,g) & =  & (\partial_n x, \tau(x).\partial_n g), \qquad i=n,
\end{eqnarray*}
where $\tau(x)$ is the {\em letter} $x$ in $GX$ and '{\bf .}' is the group product in $GX$. Let us recall
that, from the mathematical definition of $GX$, if $x$ is $n$--degenerate, for instance $x=\eta_n x'$,
the $\tau(x)$ is the null element of $GX_n$.
The behaviour of $\partial_n$ induces a change in the differential morphism of the chain complex
associated to the twisted cartesian product.

\section{The associated tensor twisted product}

We\index{tensor twisted product} recall that around the cartesian product $X \times GX$,
the Eilenberg--Zilber theorem allows to build the reduction (see the function {\tt ez}):
$$
\diagram{
{{\cal C}_*(X \times GX)} & \stackrel{h}{\longrightarrow} & {}^s{{\cal C}_*(X \times GX)} \cr
 {\scriptstyle f} \downarrow \uparrow {\scriptstyle g}  \cr
 {{\cal C}_*(X) \otimes {\cal C}_*(GX)} \cr
}
$$
On the other hand,
the differential morphism $d_\tau$ of the chain complex
${\cal C}_*(X \times_\tau GX)$ may be considered as a perturbed
differential morphism $d$ of the chain complex ${\cal C}_*(X \times GX)$,
the perturbation $\delta$ being: $\delta= d_\tau - d$. Using the
basic perturbation lemma, we may construct a new reduction:
$$
\diagram{
{{\cal C}_*(X \times_\tau GX)} & \stackrel{h}{\longrightarrow} & {}^s{{\cal C}_*(X \times_\tau GX)} \cr
 {\scriptstyle f} \downarrow \uparrow {\scriptstyle g}  \cr
 {{\cal C}_*(X) \otimes_t {\cal C}_*(GX)} \cr
}
$$
In fact, the interesting object produced by this machinery is the bottom
chain complex  ${\cal C}_*(X) \otimes_t {\cal C}_*(GX)$ which is nothing but
the {\em twisted tensor product}\footnote{{\bf Edgar H. Brown Jr.} {\em Twisted tensor products, I},
Ann. of Math., 1959, vol. 69, pp. 223-246.}.


\section{Main functions for the twisted products}

{\parindent=0mm
{\leftskip=5mm
{\tt twisted-crts-prdc} {\em space} \hfill {\em [Function]} \par}
{\leftskip=15mm
From the simplicial set $X$, here the argument {\em space},  return  the
twisted cartesian product $X \times_\tau GX$ (a simplicial set). This function calls internally the functions
{\tt loop-space} and {\tt crts-prdc} described in preceding chapters. The slot {\tt grmd} of the
resulting instance is set with the ordinary cartesian product $X \times GX$. \par}
{\leftskip=5mm
{\tt dtau-d} {\em space} \hfill {\em [Function]} \par}
{\leftskip=15mm
From the simplicial set $X$, here the argument {\em space}, build the morphism perturbation (degree $-1$)
$$d_\tau-d: X \times GX \longrightarrow X \times GX.$$
The work is essentialy done by the  lisp function {\tt dtau-d-intr} which computes the difference
between  the differential morphisms taking in account the discrepancy between
the last face operator ($\partial_n$)
in both chain complexes, $X \times_\tau GX$ and $X \times GX$. \par}
{\leftskip=5mm
{\tt szczarba} {\em space} \hfill {\em [Function]} \par}
{\leftskip=15mm
Starting from the simplicial set $X$, (the argument {\em space}), return two values: \par}
{\leftskip=20mm
1- The reduction
$$
\diagram{
{{\cal C}_*(X \times_\tau GX)} & \stackrel{h}{\longrightarrow} & {}^s{{\cal C}_*(X \times_\tau GX)} \cr
 {\scriptstyle f} \downarrow \uparrow {\scriptstyle g}  \cr
 {{\cal C}_*(X) \otimes_t {\cal C}_*(GX)} \cr
}
$$
obtained by perturbing by $d_\tau-d$ the top chain complex of the reduction
$$
\diagram{
{{\cal C}_*(X \times GX)} & \stackrel{h}{\longrightarrow} & {}^s{{\cal C}_*(X \times GX)} \cr
 {\scriptstyle f} \downarrow \uparrow {\scriptstyle g}  \cr
 {{\cal C}_*(X) \otimes {\cal C}_*(GX)} \cr
}.
$$
\par}
{\leftskip=20mm
2- The morphism corresponding to the perturbation of the bottom chain complex
induced by the perturbation $d_\tau-d$ upon the top chain complex of the above reduction. \par}
{\leftskip=15mm
This algorithm implements in particular  in a very efficient way the formulas of
Szczarba\footnote{{\bf R.H. Szczarba.} {\em The homology of twisted cartesian products.}
Transactions of the American Math. Society, 1961, vol. 100, pp. 197-216}. \par}
{\leftskip=5mm
{\tt twisted-tnsr-prdc} {\em space} \hfill {\em [Function]} \par}
{\leftskip=15mm
Return the twisted tensor product (a chain complex)  $X \otimes_t GX$.
This is done simply by extraction of the bottom chain complex from
the reduction  {\tt (szczarba {\em space})}. \par}
}
%\newpage

\subsection* {Examples}

For testing the functions above, we shall use a ``soft'' version of $\bar{\Delta}$ analogous
to {\tt soft-delta} or {\tt soft-delta-infinity}. The input of the simplices will be done
in clear with the help of the macro {\tt code}. As this version does not exists in {\tt Kenzo},
we have modified some functions of {\tt soft-delta}, as indicated.
{\footnotesize\begin{verbatim}

(DEFUN SOFT-DELTAB-CMPR (gmsm1 gmsm2)
    (if (= 1 (logcount (cdr gmsm1)))
        :EQUAL
        (f-cmpr (cdr gmsm1)(cdr gmsm2)) ))

(DEFUN SOFT-DELTAB-BNDR (dmns gmsm)
   (declare
      (fixnum dmns)
      (type soft-dlop gmsm))
   (the cmbn
      (if (< dmns 2)
         (zero-cmbn (1- dmns))
         (make-cmbn
         :degr (1- dmns)
         :list (mapcar #'(lambda (term)
                         (with-term (cffc gmsm) term
                            (term cffc (d gmsm))))
               (cmbn-list (delta-bndr dmns (cdr gmsm))))))))

(DEFUN SOFT-DELTAB ()
   (the simplicial-set
      (build-smst
         :cmpr #'soft-deltab-cmpr
         :basis :locally-effective
         :bspn (d 1)
         :face #'soft-delta-face
;;;      :intr-dgnl #'soft-delta-dgnl :dgnl-strt :gnrt
         :intr-bndr #'soft-deltab-bndr :bndr-strt :gnrt
         :orgn `(soft-deltab))))


(defmacro code (arg) `(d (dlop-ext-int ,arg)))
\end{verbatim}}
\newpage
{\footnotesize\begin{verbatim}
(setf db (soft-deltab))  ==>

[K1 Simplicial-Set]

(setf gdb (loop-space db))  ==>

[K6 Simplicial-Group]
\end{verbatim}}
Let us compare the faces of a cartesian product in $\bar{\Delta} \times \Omega(\bar{\Delta})$
and the faces of the same object in $\bar{\Delta} \times_\tau \Omega(\bar{\Delta})$:
{\footnotesize\begin{verbatim}
(setf crts-pr (crts-prdc db gdb))  ==>

[K18 Simplicial-Set]

(setf tw-crts-pr (twisted-crts-prdc db))  ==>

[K23 Simplicial-Set]

(setf smpx (crpr 0 (code '(0 1 2 3 4))
                 0 (loop3 0 (code '(0 1 2 3 4 5)) 1)))  ==>

<CrPr - 0-1-2-3-4 - <<Loop[0-1-2-3-4-5]>>>

(dotimes (j 5)
         (print (face crts-pr j 4 smpx)))  ==>

<AbSm - <CrPr - 1-2-3-4 - <<Loop[1-2-3-4-5]>>>>
<AbSm - <CrPr - 0-2-3-4 - <<Loop[0-2-3-4-5]>>>>
<AbSm - <CrPr - 0-1-3-4 - <<Loop[0-1-3-4-5]>>>>
<AbSm - <CrPr - 0-1-2-4 - <<Loop[0-1-2-4-5]>>>>
<AbSm - <CrPr - 0-1-2-3 - <<Loop[0-1-2-3-4\-1][0-1-2-3-5]>>>>

(dotimes (j 5)
         (print (face tw-crts-pr j 4 smpx)))  ==>

<AbSm - <CrPr - 1-2-3-4 - <<Loop[1-2-3-4-5]>>>>
<AbSm - <CrPr - 0-2-3-4 - <<Loop[0-2-3-4-5]>>>>
<AbSm - <CrPr - 0-1-3-4 - <<Loop[0-1-3-4-5]>>>>
<AbSm - <CrPr - 0-1-2-4 - <<Loop[0-1-2-4-5]>>>>
<AbSm - <CrPr - 0-1-2-3 - <<Loop[0-1-2-3-5]>>>>
\end{verbatim}}
\newpage
{\footnotesize\begin{verbatim}
(setf smpx2 (crpr 0 (code '(1 2 3 4 5))
                  0 (loop3 0 (code '(5 6 7 8 9 10)) 1)))  ==>

<CrPr - 1-2-3-4-5 - <<Loop[5-6-7-8-9-10]>>>

(dotimes (j 5)
         (print (face crts-pr j 4 smpx2)))  ==>

<AbSm - <CrPr - 2-3-4-5 - <<Loop[6-7-8-9-10]>>>>
<AbSm - <CrPr - 1-3-4-5 - <<Loop[5-7-8-9-10]>>>>
<AbSm - <CrPr - 1-2-4-5 - <<Loop[5-6-8-9-10]>>>>
<AbSm - <CrPr - 1-2-3-5 - <<Loop[5-6-7-9-10]>>>>
<AbSm - <CrPr - 1-2-3-4 - <<Loop[5-6-7-8-9\-1][5-6-7-8-10]>>>>

(dotimes (j 5)
         (print (face tw-crts-pr j 4 smpx2)))  ==>

<AbSm - <CrPr - 2-3-4-5 - <<Loop[6-7-8-9-10]>>>>
<AbSm - <CrPr - 1-3-4-5 - <<Loop[5-7-8-9-10]>>>>
<AbSm - <CrPr - 1-2-4-5 - <<Loop[5-6-8-9-10]>>>>
<AbSm - <CrPr - 1-2-3-5 - <<Loop[5-6-7-9-10]>>>>
<AbSm - <CrPr - 1-2-3-4 - <<Loop[1-2-3-4-5][5-6-7-8-9\-1][5-6-7-8-10]>>>>
\end{verbatim}}
The user will have noted that we may obtain the same twisted cartesian product
by the fibration theory.
Let us define a general function {\tt loop-fbr} with  a reduced simplicial set,
{\em space}, as argument. This function uses the function {\tt fibration-total}
that we have seen in the  chapter about fibrations.
Here, we see that the lisp function for the simplicial morphism, (keyword {\tt :sintr}),
implements straightforwardly the canonical twisting operator which applies a simplex of $X$
to the corresponding $1$--letter word of $GX$. Note that the execution time by
the fibration function is a little longer than for the direct twisted cartesian product
and it requires more memory.
{\footnotesize\begin{verbatim}
(defun loop-fbr (space)
             (fibration-total
                (build-smmr
                  :sorc space
                  :trgt (loop-space space)
                  :degr -1
                  :sintr #'(lambda (dmns gmsm)
                               (absm 0 (loop3 0 gmsm 1)))
                  :orgn `(total-fibration ,space))))


(setf fb-tt-db (loop-fbr db))  ==>

[K46 Simplicial-Set]

(dotimes (j 5)
         (print(face fb-tt-db j 4 smpx2)))

<AbSm - <CrPr - 2-3-4-5 - <<Loop[6-7-8-9-10]>>>>
<AbSm - <CrPr - 1-3-4-5 - <<Loop[5-7-8-9-10]>>>>
<AbSm - <CrPr - 1-2-4-5 - <<Loop[5-6-8-9-10]>>>>
<AbSm - <CrPr - 1-2-3-5 - <<Loop[5-6-7-9-10]>>>>
<AbSm - <CrPr - 1-2-3-4 - <<Loop[1-2-3-4-5][5-6-7-8-9\-1][5-6-7-8-10]>>>>

(time (dotimes (j 1000) (face tw-crts-pr 4 4 smpx2)))  ==>

; cpu time (non-gc) 143,480 msec (00:02:23.480) user, 40 msec system
; cpu time (gc)     2,140 msec user, 40 msec system
; cpu time (total)  145,620 msec (00:02:25.620) user, 80 msec system
; real time  146,899 msec (00:02:26.899)
; space allocation:
;  16,122,082 cons cells, 11,000 symbols, 6,226,992 other bytes

(time (dotimes (j 1000) (face fb-tt-db 4 4 smpx2)))  ==>

; cpu time (non-gc) 215,190 msec (00:03:35.190) user, 40 msec system
; cpu time (gc)     3,630 msec user, 50 msec system
; cpu time (total)  218,820 msec (00:03:38.820) user, 90 msec system
; real time  221,946 msec (00:03:41.946)
; space allocation:
;  23,859,004 cons cells, 17,000 symbols, 10,536,128 other bytes
\end{verbatim}}
The four following statements show the relationship
between the differentials:
{\footnotesize\begin{verbatim}
(setf perturb (dtau-d db))  ==>

[K51 Morphism (degree -1)]

(? perturb 4 smpx2)  ==>

----------------------------------------------------------------------{CMBN 3}
<1 * <CrPr - 1-2-3-4 - <<Loop[1-2-3-4-5][5-6-7-8-9\-1][5-6-7-8-10]>>>>
<-1 * <CrPr - 1-2-3-4 - <<Loop[5-6-7-8-9\-1][5-6-7-8-10]>>>>
------------------------------------------------------------------------------
\end{verbatim}}
\newpage
{\footnotesize\begin{verbatim}
(? crts-pr 4 smpx2)  ==>

----------------------------------------------------------------------{CMBN 3}
<1 * <CrPr - 1-2-3-4 - <<Loop[5-6-7-8-9\-1][5-6-7-8-10]>>>>
<-1 * <CrPr - 1-2-3-5 - <<Loop[5-6-7-9-10]>>>>
<1 * <CrPr - 1-2-4-5 - <<Loop[5-6-8-9-10]>>>>
<-1 * <CrPr - 1-3-4-5 - <<Loop[5-7-8-9-10]>>>>
<1 * <CrPr - 2-3-4-5 - <<Loop[6-7-8-9-10]>>>>
------------------------------------------------------------------------------

(? tw-crts-pr 4 smpx2)  ==>

----------------------------------------------------------------------{CMBN 3}
<1 * <CrPr - 1-2-3-4 - <<Loop[1-2-3-4-5][5-6-7-8-9\-1][5-6-7-8-10]>>>>
<-1 * <CrPr - 1-2-3-5 - <<Loop[5-6-7-9-10]>>>>
<1 * <CrPr - 1-2-4-5 - <<Loop[5-6-8-9-10]>>>>
<-1 * <CrPr - 1-3-4-5 - <<Loop[5-7-8-9-10]>>>>
<1 * <CrPr - 2-3-4-5 - <<Loop[6-7-8-9-10]>>>>
------------------------------------------------------------------------------
\end{verbatim}}
The same type of tests may be done with another reduced simplicial set, namely $P^3{\R}$.
{\footnotesize\begin{verbatim}
(setf pri3 (R-proj-space 3))  ==>

[K52 Simplicial-Set]

(setf tw-pri3 (twisted-crts-prdc pri3))  ==>

[K74 Simplicial-Set]

(basis tw-pri3)  ==>

:LOCALLY-EFFECTIVE

(setf crt-pri3 (crts-prdc pri3 (loop-space pri3)))  ==>

[K69 Simplicial-Set]

(setf s (crpr 0 4 0 (loop3 0 5 1)))  ==>

<CrPr - 4 - <<Loop[5]>>>
\end{verbatim}}
\newpage
{\footnotesize\begin{verbatim}
(dotimes (i 5) (print (face crt-pri3 i 4 s)))  ==>

<AbSm - <CrPr - 3 - <<Loop[4]>>>>
<AbSm 0 <CrPr 1-0 0 - <<Loop[3]>>>>
<AbSm 1 <CrPr 1-0 0 - <<Loop[3]>>>>
<AbSm 2 <CrPr 1-0 0 - <<Loop[3]>>>>
<AbSm - <CrPr - 3 - <<Loop[4\-1]>>>>

(dotimes (i 5) (print (face tw-pri3 i 4 s)))  ==>

<AbSm - <CrPr - 3 - <<Loop[4]>>>>
<AbSm 0 <CrPr 1-0 0 - <<Loop[3]>>>>
<AbSm 1 <CrPr 1-0 0 - <<Loop[3]>>>>
<AbSm 2 <CrPr 1-0 0 - <<Loop[3]>>>>
<AbSm - <CrPr - 3 2-1-0 <<Loop>>>>

(setf fb-tt-pri3 (loop-fbr pri3))  ==>

[K80 Simplicial-Set]

(dotimes (j 5) (print (face fb-tt-pri3 j 4 s)))

<AbSm - <CrPr - 3 - <<Loop[4]>>>>
<AbSm 0 <CrPr 1-0 0 - <<Loop[3]>>>>
<AbSm 1 <CrPr 1-0 0 - <<Loop[3]>>>>
<AbSm 2 <CrPr 1-0 0 - <<Loop[3]>>>>
<AbSm - <CrPr - 3 2-1-0 <<Loop>>>>
\end{verbatim}}
Let us check now the Szczarba reduction, with the simplicial set $\bar{\Delta}$.
{\footnotesize\begin{verbatim}
(setf szc-reduc (szczarba db))  ==>

[K107 Reduction]

(pre-check-rdct szc-reduc)  ==>

---done---

(setf *tc* (cmbn 2 1 (crpr 0 (code '(0 1 2))
                           0 (loop3 0 (code '(5 6 7 8)) 1))))

----------------------------------------------------------------------{CMBN 2}
<1 * <CrPr - 0-1-2 - <<Loop[5-6-7-8]>>>>
------------------------------------------------------------------------------

(setf *bc* (cmbn 2 1 (tnpr 0 (code '(0)) 2 (loop3 0 (code '(5 6 7 8)) 1))
                  10 (tnpr 1 (code '(10 11)) 1 (loop3 0 (code '(0 1 2)) 1))
                 100 (tnpr 2 (code '(0 1 2)) 0 (loop3 0 (code '(10 11)) 1))))

----------------------------------------------------------------------{CMBN 2}
<1 * <TnPr 0 <<Loop[5-6-7-8]>>>>
<10 * <TnPr 10-11 <<Loop[0-1-2]>>>>
<100 * <TnPr 0-1-2 <<Loop[10-11]>>>>
------------------------------------------------------------------------------

(pre-check-rdct szc-reduc)  ==>

---done---

(check-rdct)  ==>

*TC* =>
----------------------------------------------------------------------{CMBN 2}
<1 * <CrPr - 0-1-2 - <<Loop[5-6-7-8]>>>>
------------------------------------------------------------------------------

*BC* =>
----------------------------------------------------------------------{CMBN 2}
<1 * <TnPr 0 <<Loop[5-6-7-8]>>>>
<10 * <TnPr 10-11 <<Loop[0-1-2]>>>>
<100 * <TnPr 0-1-2 <<Loop[10-11]>>>>
------------------------------------------------------------------------------

Checking *TDD* = 0
Result:
----------------------------------------------------------------------{CMBN 0}
------------------------------------------------------------------------------

              ....... All results null ......

Checking *HG* = 0
Result:
----------------------------------------------------------------------{CMBN 3}
------------------------------------------------------------------------------

---done---
\end{verbatim}}

\newpage

\section{Exercice with the twisting cochain.}

Let $B$ be a reduced simplicial set and its loop space $GB$ (a simplicial group).
In the following examples, we shall use the efficient {\tt Kenzo}
version of the simplicial set $\bar{\Delta}$, i.e. the simplices are input
and printed under their binary form coding.

\subsection {The twisting cochain.}

The\index{twisting cochain} twisting cochain $t$ is the morphism
$$t: aug \circ d_{\otimes t} \circ coaug$$
where,
\begin{enumerate}
\item $coaug$ is the morphism
$$ coaug: B \rightarrow B \otimes GB,$$
$$ coaug(b) = b \otimes 1_{GB}. $$
\item $d_{\otimes t}$ is the differential morphism of the twisted tensor product
$B \otimes_t GB$.
\item $aug$ is the morphism
$$aug: B \otimes GB  \longrightarrow GB ,  $$
\begin{eqnarray*}
aug(b_i \otimes g_j)&  = & g_j, \qquad b_i \in B_0, \\
aug(b_i \otimes g_j)&  = & 0, \qquad b_i \in B_k,\, k\not=0.
\end{eqnarray*}
\end{enumerate}
We recall that, as graded modules, $B\otimes GB$ and $B \otimes_t GB$ are
identical.
\par
First, we create $\bar{\Delta}$ and $G\bar{\Delta}$. The function {\tt twisted-tnsr-prdc}
extracts the twisted tensor product $\bar{\Delta} \otimes_t G\bar{\Delta}$ from
the reduction obtained by a call to the function {\tt szczarba}.
{\footnotesize\begin{verbatim}

(setf d (deltab)) ==>

[K1 Simplicial-Set]

(setf gd (loop-space d))  ==>

[K6 Simplicial-Group]

(setf d-twtp-gd (twisted-tnsr-prdc d))  ==>

[K49 Chain-Complex]
\end{verbatim}}
Let us define the  coaugmentation and augmentation morphisms. As usual, it is sufficient
to define the algorithm on a generator.
{\footnotesize\begin{verbatim}
(setf coaug (build-mrph
               :sorc d
               :trgt d-twtp-gd
               :degr 0
               :intr #'(lambda (degr gnrt)
                         (term-cmbn degr 1 (tnpr degr gnrt 0 +null-loop+)))
               :strt :gnrt
               :orgn '(coaug. d --> d-twtp-gd)))  ==>

[K52 Morphism (degree 0)]

(? coaug 3 15)  ==>

----------------------------------------------------------------------{CMBN 3}
<1 * <TnPr 15 <<Loop>>>>
------------------------------------------------------------------------------

(? coaug (cmbn 4 1 31 1 62 1 124))  ==>

----------------------------------------------------------------------{CMBN 4}
<1 * <TnPr 31 <<Loop>>>>
<1 * <TnPr 62 <<Loop>>>>
<1 * <TnPr 124 <<Loop>>>>
------------------------------------------------------------------------------

(setf aug (build-mrph
             :sorc d-twtp-gd
             :trgt gd
             :degr 0
             :intr #'(lambda (degr tnpr)
                        (with-tnpr (degr1 gmsm1 degr2 loop2) tnpr
                            (if (zerop degr1)
                                (term-cmbn degr 1 loop2)
                                (zero-cmbn degr))))
             :strt :gnrt
             :orgn '(aug. of d-twtp-gd)))  ==>

[K53 Morphism (degree 0)]

(? aug (cmbn 2 9 (tnpr 0 1 2 (loop3 0 15 1))
               3 (tnpr 1 3 1 (loop3 0 7 1))))  ==>

----------------------------------------------------------------------{CMBN 2}
<9 * <<Loop[15]>>>
------------------------------------------------------------------------------
\end{verbatim}}
The morphism {\tt cochain} is simply a composition of morphisms.
We apply it upon the simplices {\tt (0)},
{\tt (0 1 2)}, {\tt (0 1 2 3)}, ...,
coded as the integers $1,7,15,\ldots$.
{\footnotesize\begin{verbatim}
(setf cochain (i-cmps aug (dffr d-twtp-gd) coaug))  ==>

[K55 Morphism (degree -1)]

(? cochain 0 1)  ==>

----------------------------------------------------------------------{CMBN -1}
------------------------------------------------------------------------------

(? cochain 2 7)  ==>

----------------------------------------------------------------------{CMBN 1}
<1 * <<Loop[0 3][7]>>>
------------------------------------------------------------------------------

(? cochain 3 15)  ==>

----------------------------------------------------------------------{CMBN 2}
<1 * <<Loop[1-0 3][0 7][1 13]>>>
<-1 * <<Loop[1-0 3][1 7][15]>>>
------------------------------------------------------------------------------

(? cochain 4 31)  ==>

----------------------------------------------------------------------{CMBN 3}
<-1 * <<Loop[2-1-0 3][1-0 7][2-0 13][2-1 25]>>>
<1 * <<Loop[2-1-0 3][1-0 7][2-1 13][2 29]>>>
<1 * <<Loop[2-1-0 3][2-0 7][0 15][2-1 25]>>>
<-1 * <<Loop[2-1-0 3][2-0 7][2-1 13][1 29]>>>
<-1 * <<Loop[2-1-0 3][2-1 7][1 15][2 27]>>>
<1 * <<Loop[2-1-0 3][2-1 7][2 15][31]>>>
------------------------------------------------------------------------------

(? cochain 5 63)  ==>

----------------------------------------------------------------------{CMBN 4}
<-1 * <<Loop[3-2-1-0 3][2-1-0 7][3-1-0 13][3-2-0 25][3-2-1 49]>>>
<1 * <<Loop[3-2-1-0 3][2-1-0 7][3-1-0 13][3-2-1 25][3-2 57]>>>
<1 * <<Loop[3-2-1-0 3][2-1-0 7][3-2-0 13][3-0 29][3-2-1 49]>>>
<-1 * <<Loop[3-2-1-0 3][2-1-0 7][3-2-0 13][3-2-1 25][3-1 57]>>>
<-1 * <<Loop[3-2-1-0 3][2-1-0 7][3-2-1 13][3-1 29][3-2 53]>>>
<1 * <<Loop[3-2-1-0 3][2-1-0 7][3-2-1 13][3-2 29][3 61]>>>
<1 * <<Loop[3-2-1-0 3][3-1-0 7][1-0 15][3-2-0 25][3-2-1 49]>>>
<-1 * <<Loop[3-2-1-0 3][3-1-0 7][1-0 15][3-2-1 25][3-2 57]>>>
<-1 * <<Loop[3-2-1-0 3][3-1-0 7][3-2-0 13][2-0 29][3-2-1 49]>>>
<1 * <<Loop[3-2-1-0 3][3-1-0 7][3-2-0 13][3-2-1 25][2-1 57]>>>
<1 * <<Loop[3-2-1-0 3][3-1-0 7][3-2-1 13][2-1 29][3-2 53]>>>
<-1 * <<Loop[3-2-1-0 3][3-1-0 7][3-2-1 13][3-2 29][2 61]>>>
<-1 * <<Loop[3-2-1-0 3][3-2-0 7][2-0 15][3-0 27][3-2-1 49]>>>
<1 * <<Loop[3-2-1-0 3][3-2-0 7][2-0 15][3-2-1 25][3-1 57]>>>
<1 * <<Loop[3-2-1-0 3][3-2-0 7][3-0 15][0 31][3-2-1 49]>>>
<-1 * <<Loop[3-2-1-0 3][3-2-0 7][3-0 15][3-2-1 25][2-1 57]>>>
<-1 * <<Loop[3-2-1-0 3][3-2-0 7][3-2-1 13][2-1 29][3-1 53]>>>
<1 * <<Loop[3-2-1-0 3][3-2-0 7][3-2-1 13][3-1 29][1 61]>>>
<1 * <<Loop[3-2-1-0 3][3-2-1 7][2-1 15][3-1 27][3-2 51]>>>
<-1 * <<Loop[3-2-1-0 3][3-2-1 7][2-1 15][3-2 27][3 59]>>>
<-1 * <<Loop[3-2-1-0 3][3-2-1 7][3-1 15][1 31][3-2 51]>>>
<1 * <<Loop[3-2-1-0 3][3-2-1 7][3-1 15][3-2 27][2 59]>>>
<1 * <<Loop[3-2-1-0 3][3-2-1 7][3-2 15][2 31][3 55]>>>
<-1 * <<Loop[3-2-1-0 3][3-2-1 7][3-2 15][3 31][63]>>>
------------------------------------------------------------------------------
\end{verbatim}}
We verify, that for a simplex in dimension $n$, the length of the resulting
combination is $(n-1)!$.
{\footnotesize\begin{verbatim}
(length(cmbn-list *))  ==>

24

(? cochain 6 127)  ==>

----------------------------------------------------------------------{CMBN 5}
<1 * <<Loop[4-3-2-1-0 3][3-2-1-0 7][4-2-1-0 13][4-3-1-0 25][4-3-2-0 49][4-3-2-1 97]>>>
<-1 * <<Loop[4-3-2-1-0 3][3-2-1-0 7][4-2-1-0 13][4-3-1-0 25][4-3-2-1 49][4-3-2 113]>>>
<-1 * <<Loop[4-3-2-1-0 3][3-2-1-0 7][4-2-1-0 13][4-3-2-0 25][4-3-0 57][4-3-2-1 97]>>>
<1 * <<Loop[4-3-2-1-0 3][3-2-1-0 7][4-2-1-0 13][4-3-2-0 25][4-3-2-1 49][4-3-1 113]>>>
<1 * <<Loop[4-3-2-1-0 3][3-2-1-0 7][4-2-1-0 13][4-3-2-1 25][4-3-1 57][4-3-2 105]>>>
<-1 * <<Loop[4-3-2-1-0 3][3-2-1-0 7][4-2-1-0 13][4-3-2-1 25][4-3-2 57][4-3 121]>>>
<-1 * <<Loop[4-3-2-1-0 3][3-2-1-0 7][4-3-1-0 13][4-1-0 29][4-3-2-0 49][4-3-2-1 97]>>>
<1 * <<Loop[4-3-2-1-0 3][3-2-1-0 7][4-3-1-0 13][4-1-0 29][4-3-2-1 49][4-3-2 113]>>>
<1 * <<Loop[4-3-2-1-0 3][3-2-1-0 7][4-3-1-0 13][4-3-2-0 25][4-2-0 57][4-3-2-1 97]>>>
<-1 * <<Loop[4-3-2-1-0 3][3-2-1-0 7][4-3-1-0 13][4-3-2-0 25][4-3-2-1 49][4-2-1 113]>>>
<-1 * <<Loop[4-3-2-1-0 3][3-2-1-0 7][4-3-1-0 13][4-3-2-1 25][4-2-1 57][4-3-2 105]>>>
<1 * <<Loop[4-3-2-1-0 3][3-2-1-0 7][4-3-1-0 13][4-3-2-1 25][4-3-2 57][4-2 121]>>>
... ...
------------------------------------------------------------------------------

(length(cmbn-list *))  ==>

120
\end{verbatim}}

\subsection {Expression of the differential $d_{\otimes t}$}

The differential $d_{\otimes t}$ of the twisted tensor product $B \otimes_t GB$ may be expressed as the sum
of the differential of the simple tensor product $B \otimes GB$ and the following morphism of degree $-1$,
$$\nabla: B \otimes GB \longrightarrow B \otimes GB,$$
where $\nabla$ is the composition of the morphisms of the following sequence:
$$B\otimes GB \buildrel {\Delta\otimes 1_{GB}} \over \longrightarrow  (B\otimes B) \otimes GB
              \buildrel {assoc} \over \longrightarrow  B \otimes (B \otimes GB)
              \buildrel {1_B \otimes (t \otimes 1_{GB})} \over \longrightarrow  B \otimes (GB \otimes GB)
              \buildrel {1_B \otimes \varpi} \over \longrightarrow B \otimes GB.$$
In the previous formula, $\Delta$ is the coproduct in the coalgebra $B$, $\varpi$ is the
product in the algebra $GB$, $1_B$ is the identity morphism in $B$ and $1_{GB}$ the identity morphism
in $GB$. We need also, as in a previous example, an ``assoc'' morphism for
the compound tensor product. The user will note the easiness for coding in {\tt Kenzo} such
a long composition of morphisms.
{\footnotesize\begin{verbatim}

(setf 3-left (tnsr-prdc (tnsr-prdc d d) gd))  ==>

[K56 Chain-Complex]

(setf 3-right (tnsr-prdc d (tnsr-prdc d gd)))  ==>

[K58 Chain-Complex]

(setf assoc (build-mrph
               :sorc 3-left
               :trgt 3-right
               :degr 0
               :intr #'(lambda (degr a2-a)
                         (with-tnpr (degra2 gnrta2 degra gnrta) a2-a
                            (with-tnpr (degr1 gnrt1 degr2 gnrt2) gnrta2
                               (cmbn (+ degr1 degr2 degra)
                                     1 (tnpr degr1
                                             gnrt1
                                             (+ degr2 degra)
                                             (tnpr degr2 gnrt2 degra gnrta))))))
               :strt :gnrt
               :orgn '(assoc-double-tensor-product)))   ==>

[K60 Morphism (degree 0)]

(setf id-d (idnt-mrph d))  ==>

[K61 Morphism (degree 0)]

(setf id-gd (idnt-mrph gd))  ==>

[K62 Morphism (degree 0)]

(setf nabla
 (add (dffr (tnsr-prdc d gd))
      (i-cmps (tnsr-prdc id-d (aprd gd))
              (tnsr-prdc id-d (tnsr-prdc cochain id-gd))
              assoc
              (tnsr-prdc (cprd d) id-gd))))  ==>

[K74 Morphism (degree -1)]
\end{verbatim}}
We verify that the morphism {\tt nabla} and the differential of the
twisted tensor product (accessible
through the symbol {\tt d-twtp-gd}) return the same results:
{\footnotesize\begin{verbatim}
(? (dffr d-twtp-gd) 2 (tnpr 1 3 1 (loop3 0 7 1)))  ==>

----------------------------------------------------------------------{CMBN 1}
<1 * <TnPr 1 <<Loop[7]>>>>
<-1 * <TnPr 1 <<Loop[0 3][7]>>>>
<1 * <TnPr 3 <<Loop[3\-1][5]>>>>
<-1 * <TnPr 3 <<Loop[6]>>>>
------------------------------------------------------------------------------

(? nabla 2 (tnpr 1 3 1 (loop3 0 7 1)))  ==>

----------------------------------------------------------------------{CMBN 1}
<1 * <TnPr 1 <<Loop[7]>>>>
<-1 * <TnPr 1 <<Loop[0 3][7]>>>>
<1 * <TnPr 3 <<Loop[3\-1][5]>>>>
<-1 * <TnPr 3 <<Loop[6]>>>>
------------------------------------------------------------------------------

(setf zero (sbtr nabla (dffr d-twtp-gd)))  ==>

[K75 Morphism (degree -1)]

(? zero 2 (tnpr 1 3 1 (loop3 0 7 1)))  ==>

----------------------------------------------------------------------{CMBN 1}
------------------------------------------------------------------------------

(? zero 5 (tnpr  3 15 2 (loop3 0 15 1)))  ==>

----------------------------------------------------------------------{CMBN 4}
------------------------------------------------------------------------------
\end{verbatim}}

\newpage

\subsection {The cup product of the twisting cochain}

The cup product\index{twisting cochain!cup product} $t \sqcup t$ of the twisting cochain $t$,
$$t\sqcup t: B \longrightarrow GB$$
is defined by the following composition of morphisms:
$$B \buildrel {\Delta} \over \longrightarrow B \otimes B
    \buildrel {t \otimes t} \over \longrightarrow GB \otimes GB
    \buildrel {\varpi} \over \longrightarrow GB.$$
We may now verify the formula
$$ d_{GB} \circ t + t \circ d_B + t\sqcup t =0,$$
where $d_B$ and $d_{GB}$ are  the respective differential morphisms in $B$ and $GB$.
{\footnotesize\begin{verbatim}

(setf t-cup-t (i-cmps (aprd gd) (tnsr-prdc cochain cochain) (cprd d)))  ==>

[K78 Morphism (degree -2)]

(setf dt (cmps (dffr gd) cochain))  ==>

[K79 Morphism (degree -2)]

(setf td (cmps cochain (dffr d)))  ==>

[K80 Morphism (degree -2)]

(setf zero-2 (i-add dt td t-cup-t))  ==>

[K82 Morphism (degree -2)]

(? zero-2 3 15)  ==>

----------------------------------------------------------------------{CMBN 1}
------------------------------------------------------------------------------

(? zero-2 4 31)  ==>

----------------------------------------------------------------------{CMBN 2}
------------------------------------------------------------------------------

(? zero-2 5 63)  ==>

----------------------------------------------------------------------{CMBN 3}
------------------------------------------------------------------------------

(? zero-2 6 127)  ==>

----------------------------------------------------------------------{CMBN 4}
------------------------------------------------------------------------------
\end{verbatim}}

So the user has easily verified the {\em cochain condition} for the twisting
cochain $t$.
\newpage

\section {The essential contraction}

It is known that if $B$ is a reduced simplicial set, the space $B \times_\tau GB$ is
contractible. So it is possible to build a reduction of this space over $\mathbb{Z}$.
This reduction depends of the important contraction
$$\chi_{\times\tau}:  {{\cal C}_*(B) \times_\tau {\cal C}_*(GB)} \longrightarrow
                      {{\cal C}_*(B) \times_\tau {\cal C}_*(GB)}, $$
which is a homotopy operator (degree: $+1$) satisfying the relation
$$d \circ \chi_{\times\tau} + \chi_{\times\tau} \circ d =1.$$
Now, from the reduction
$$
\diagram{
{{\cal C}_*(X \times_\tau GX)} & \stackrel{h}{\longrightarrow} & {}^s{{\cal C}_*(X \times_\tau GX)} \cr
 {\scriptstyle f} \downarrow \uparrow {\scriptstyle g}  \cr
 {{\cal C}_*(X) \otimes_t {\cal C}_*(GX)} \cr
}
$$
there is also  an induced contraction
$$\chi_{\otimes t}:  {{\cal C}_*(B) \otimes_t {\cal C}_*(GB)} \longrightarrow
                      {{\cal C}_*(B) \otimes_t {\cal C}_*(GB)}, $$
defined by $\chi_{\otimes t}= f \circ \chi_{\times\tau} \circ g$.
This implies that the twisted tensor product is also contractible over $\mathbb{Z}$.
\par
The two functions of {\tt Kenzo} for this construction are:
\vskip 0.35cm
{\parindent=0mm
{\leftskip=5mm
{\tt crts-contraction} {\em space} \hfill {\em [Function]} \par}
{\leftskip=15mm
Return the homotopy morphism corresponding to the contraction
$\chi_{\times\tau}$. The lisp definition is:
{\footnotesize\begin{verbatim}
          (defun CRTS-CONTRACTION
             (space
                &aux (twisted-crts-prdc (twisted-crts-prdc space)))
             (the morphism
                (build-mrph
                   :sorc twisted-crts-prdc
                   :trgt twisted-crts-prdc
                   :degr +1
                   :intr (crts-contraction-intr
                               (cmpr space)
                               (bspn space)
                               (face space)
                               (cmpr twisted-crts-prdc))
                   :strt :gnrt
                   :orgn `(crts-contraction ,space))))
\end{verbatim}}
At execution time, the work is essentialy done by the function put in the
slot {\tt :intr}. \par}
{\leftskip=5mm
{\tt tnpr-contraction} {\em space} \hfill {\em [Function]} \par}
{\leftskip=15mm
Return the induced morphism $\chi_{\otimes t}= f \circ \chi_{\times\tau} \circ g$,
as shown by the lisp definition:
{\footnotesize\begin{verbatim}
          (defun TNPR-CONTRACTION
             (space
                &aux (szczarba (szczarba space))
                     (f (f szczarba))
                     (g (g szczarba))
                     (crts-contraction (crts-contraction space)))

             (the morphism
                (i-cmps f crts-contraction g)))
\end{verbatim}} \par}
}

\subsection* {Examples}

Let us use again our familiar simplicial set $\bar\Delta$. First we apply  the
homotopy operator $\chi_{\times t}$   upon various generators.
Note that the binary coding of the simplices
allows normal arithmetic operators to generate them.
{\footnotesize\begin{verbatim}
(setf delta (deltab))  ==>

[K1 Simplicial-Set]

(setf chi-x (crts-contraction delta))  ==>

[K28 Morphism (degree 1): K23 -> K23]

(? chi-x 0 (crpr 0 1 0 +null-loop+))  ==>

----------------------------------------------------------------------{CMBN 1}
------------------------------------------------------------------------------

(? chi-x 0 (crpr 0 1 0 (loop3 0 96 1)))  ==>

----------------------------------------------------------------------{CMBN 1}
<-1 * <CrPr - 96 0 <<Loop>>>>
------------------------------------------------------------------------------

(? chi-x 0 (crpr 0 1 0 (loop3 0 96 1 0 (+ 256 128) 1)))  ==>

----------------------------------------------------------------------{CMBN 1}
<-1 * <CrPr - 96 0 <<Loop[384]>>>>
<-1 * <CrPr - 384 0 <<Loop>>>>
------------------------------------------------------------------------------

(? chi-x 0 (crpr 0 1 0 (loop3 0 96 1 0 (+ 256 128) 1 0 (+ 512 1024) 1)))  ==>

----------------------------------------------------------------------{CMBN 1}
<-1 * <CrPr - 96 0 <<Loop[384][1536]>>>>
<-1 * <CrPr - 384 0 <<Loop[1536]>>>>
<-1 * <CrPr - 1536 0 <<Loop>>>>
------------------------------------------------------------------------------

(? chi-x 0 (crpr 0 1 0 (loop3 0 96 2)))  ==>

----------------------------------------------------------------------{CMBN 1}
<-1 * <CrPr - 96 0 <<Loop>>>>
<-1 * <CrPr - 96 0 <<Loop[96]>>>>
------------------------------------------------------------------------------

(? chi-x 1 (crpr 0 3 0 (loop3 0 (+ 32 64 128) 1)))  ==>

----------------------------------------------------------------------{CMBN 2}
<1 * <CrPr - 224 1-0 <<Loop>>>>
<-1 * <CrPr 1 3 0 <<Loop[224]>>>>
<1 * <CrPr 1 96 0 <<Loop[224]>>>>
------------------------------------------------------------------------------

(? chi-x 2 (crpr 0 7 0 (loop3 0 (+ 32 64 128 256) 1)))  ==>

----------------------------------------------------------------------{CMBN 3}
<-1 * <CrPr - 480 2-1-0 <<Loop>>>>
<1 * <CrPr 2 7 1 <<Loop[480]>>>>
<-1 * <CrPr 2 224 1 <<Loop[480]>>>>
<-1 * <CrPr 2-1 5 0 <<Loop[480]>>>>
<1 * <CrPr 2-1 160 0 <<Loop[480]>>>>
------------------------------------------------------------------------------

(? chi-x 3 (crpr 2 7 1 (loop3 2 (+ 32 64 128) 2 4 (+ 32 64 128) -2)))  ==>

----------------------------------------------------------------------{CMBN 4}
<-1 * <CrPr 3-1 7 2-0 <<Loop[1 224\2][2 224\-2]>>>>
<1 * <CrPr 3-2 7 1-0 <<Loop[1 224\2][2 224\-2]>>>>
------------------------------------------------------------------------------

(? chi-x 3 (crpr 2 7 4 (loop3 2 (+ 32 64 128) 2 1 (+ 32 64 128) -2)))  ==>

----------------------------------------------------------------------{CMBN 4}
------------------------------------------------------------------------------

(? chi-x 3 (crpr 2 7 1 (loop3 2 (+ 32 64 128) -2 4 (+ 32 64 128) 2)))  ==>

----------------------------------------------------------------------{CMBN 4}
<-1 * <CrPr 3-1 7 2-0 <<Loop[1 224\-2][2 224\2]>>>>
<1 * <CrPr 3-2 7 1-0 <<Loop[1 224\-2][2 224\2]>>>>
------------------------------------------------------------------------------

(? chi-x 3 (crpr 2 7 4 (loop3 2 (+ 32 64 128) -2 1 (+ 32 64 128) 2)))  ==>

----------------------------------------------------------------------{CMBN 4}
------------------------------------------------------------------------------

(? chi-x 3 (crpr 1 7 2 (loop3 2 (+ 32 64 128) 2 4 (+ 32 64 128) -2)))  ==>

----------------------------------------------------------------------{CMBN 4}
<-1 * <CrPr 3-0 7 2-1 <<Loop[1 224\2][2 224\-2]>>>>
------------------------------------------------------------------------------

(? chi-x 3 (crpr 4 7 2 (loop3 2 (+ 32 64 128) 2 1 (+ 32 64 128) -2)))  ==>

----------------------------------------------------------------------{CMBN 4}
------------------------------------------------------------------------------

(? chi-x 3 (crpr 1 7 2 (loop3 2 (+ 32 64 128) -2 4 (+ 32 64 128) 2)))  ==>

----------------------------------------------------------------------{CMBN 4}
<-1 * <CrPr 3-0 7 2-1 <<Loop[1 224\-2][2 224\2]>>>>
------------------------------------------------------------------------------

(? chi-x 3 (crpr 4 7 2 (loop3 2 (+ 32 64 128) -2 1 (+ 32 64 128) 2)))  ==>

----------------------------------------------------------------------{CMBN 4}
------------------------------------------------------------------------------
\end{verbatim}}
Then, for testing  $\chi_{\otimes t}$
we use the image of all those generators by the function {\tt f} of the Brown reduction.
$$
\diagram{
{{\cal C}_*(X \times_\tau GX)} & \stackrel{h}{\longrightarrow} & {}^s{{\cal C}_*(X \times_\tau GX)} \cr
 {\scriptstyle f} \downarrow \uparrow {\scriptstyle g}  \cr
 {{\cal C}_*(X) \otimes_t {\cal C}_*(GX)} \cr
}
$$
{\footnotesize\begin{verbatim}
(setf *tnpr-with-degrees* t)  ==>

T

(setf chi-t (tnpr-contraction delta))  ==>

[K54 Morphism (degree 1): K50 -> K50]

(setf reduc (szczarba delta))  ==>

[K52 Reduction]

(setf f (f reduc))  ==>

[K42 Morphism (degree 0): K23 -> K50]

(? f 0 (crpr 0 1 0 +null-loop+))  ==>

----------------------------------------------------------------------{CMBN 0}
<1 * <TnPr 0 1 0 <<Loop>>>>
------------------------------------------------------------------------------

(? chi-t *)  ==>

----------------------------------------------------------------------{CMBN 1}
------------------------------------------------------------------------------

(? f 0 (crpr 0 1 0 (loop3 0 96 1))) ==>

----------------------------------------------------------------------{CMBN 0}
<1 * <TnPr 0 1 0 <<Loop[96]>>>>
------------------------------------------------------------------------------
\end{verbatim}}
\newpage
{\footnotesize\begin{verbatim}
(? chi-t *)  ==>

----------------------------------------------------------------------{CMBN 1}
<-1 * <TnPr 1 96 0 <<Loop>>>>
------------------------------------------------------------------------------

(? f 0 (crpr 0 1 0 (loop3 0 96 1 0 (+ 256 128) 1)))  ==>

----------------------------------------------------------------------{CMBN 0}
<1 * <TnPr 0 1 0 <<Loop[96][384]>>>>
------------------------------------------------------------------------------

(? chi-t *)  ==>

----------------------------------------------------------------------{CMBN 1}
<-1 * <TnPr 1 96 0 <<Loop[384]>>>>
<-1 * <TnPr 1 384 0 <<Loop>>>>
------------------------------------------------------------------------------

(? f 0 (crpr 0 1 0 (loop3 0 96 1 0 (+ 256 128) 1 0 (+ 512 1024) 1)))  ==>

----------------------------------------------------------------------{CMBN 0}
<1 * <TnPr 0 1 0 <<Loop[96][384][1536]>>>>
------------------------------------------------------------------------------

(? chi-t *)  ==>

----------------------------------------------------------------------{CMBN 1}
<-1 * <TnPr 1 96 0 <<Loop[384][1536]>>>>
<-1 * <TnPr 1 384 0 <<Loop[1536]>>>>
<-1 * <TnPr 1 1536 0 <<Loop>>>>
------------------------------------------------------------------------------

(? f 0 (crpr 0 1 0 (loop3 0 96 2)))  ==>

----------------------------------------------------------------------{CMBN 0}
<1 * <TnPr 0 1 0 <<Loop[96\2]>>>>
------------------------------------------------------------------------------

(? chi-t *)  ==>

----------------------------------------------------------------------{CMBN 1}
<-1 * <TnPr 1 96 0 <<Loop>>>>
<-1 * <TnPr 1 96 0 <<Loop[96]>>>>
------------------------------------------------------------------------------
\end{verbatim}}
\newpage
{\footnotesize\begin{verbatim}
(? f 1 (crpr 0 3 0 (loop3 0 (+ 32 64 128) 1)))  ==>

----------------------------------------------------------------------{CMBN 1}
<1 * <TnPr 0 1 1 <<Loop[0 3][224]>>>>
<1 * <TnPr 1 3 0 <<Loop[192]>>>>
------------------------------------------------------------------------------

(? chi-t *)  ==>

----------------------------------------------------------------------{CMBN 2}
<-1 * <TnPr 1 3 1 <<Loop[224]>>>>
<1 * <TnPr 1 96 1 <<Loop[224]>>>>
<1 * <TnPr 2 224 0 <<Loop>>>>
------------------------------------------------------------------------------

(? f 2 (crpr 0 7 0 (loop3 0 (+ 32 64 128 256) 1)))  ==>

----------------------------------------------------------------------{CMBN 2}
<-1 * <TnPr 0 1 2 <<Loop[1-0 3][0 7][1 416]>>>>
<1 * <TnPr 0 1 2 <<Loop[1-0 3][1 7][480]>>>>
<1 * <TnPr 1 3 1 <<Loop[0 6][448]>>>>
<1 * <TnPr 2 7 0 <<Loop[384]>>>>
------------------------------------------------------------------------------

(? chi-t *)  ==>

----------------------------------------------------------------------{CMBN 3}
<1 * <TnPr 0 1 3 <<Loop[2-1-0 3][1-0 7][2 480]>>>>
<-1 * <TnPr 0 1 3 <<Loop[2-1-0 3][2-0 7][1 480]>>>>
<-1 * <TnPr 0 32 3 <<Loop[2-1-0 96][1-0 224][2 480]>>>>
<1 * <TnPr 0 32 3 <<Loop[2-1-0 96][2-0 224][1 480]>>>>
<-1 * <TnPr 1 5 2 <<Loop[480]>>>>
<1 * <TnPr 1 160 2 <<Loop[480]>>>>
<1 * <TnPr 2 7 1 <<Loop[448]>>>>
<-1 * <TnPr 2 224 1 <<Loop[448]>>>>
<-1 * <TnPr 3 480 0 <<Loop>>>>
------------------------------------------------------------------------------

(? f 3 (crpr 2 7 1 (loop3 2 (+ 32 64 128) 2 4 (+ 32 64 128) -2)))  ==>

----------------------------------------------------------------------{CMBN 3}
<1 * <TnPr 0 1 3 <<Loop[2-1-0 3][2-1 7][2-0 224\2][3-0 224\-2]>>>>
------------------------------------------------------------------------------
\end{verbatim}}
\newpage
{\footnotesize\begin{verbatim}
(? chi-t *)  ==>

----------------------------------------------------------------------{CMBN 4}
<1 * <TnPr 2 7 2 <<Loop[1 224\2][2 224\-2]>>>>
------------------------------------------------------------------------------

(? f 3 (crpr 2 7 4 (loop3 2 (+ 32 64 128) 2 1 (+ 32 64 128) -2)))  ==>

----------------------------------------------------------------------{CMBN 3}
------------------------------------------------------------------------------

(? chi-t *)  ==>

----------------------------------------------------------------------{CMBN 4}
------------------------------------------------------------------------------

(? f 3 (crpr 2 7 1 (loop3 2 (+ 32 64 128) -2 4 (+ 32 64 128) 2)))  ==>

----------------------------------------------------------------------{CMBN 3}
<1 * <TnPr 0 1 3 <<Loop[2-1-0 3][2-1 7][2-0 224\-2][3-0 224\2]>>>>
------------------------------------------------------------------------------

(? chi-t *)  ==>

----------------------------------------------------------------------{CMBN 4}
<1 * <TnPr 2 7 2 <<Loop[1 224\-2][2 224\2]>>>>
------------------------------------------------------------------------------

(? f 3 (crpr 2 7 4 (loop3 2 (+ 32 64 128) -2 1 (+ 32 64 128) 2)))  ==>

----------------------------------------------------------------------{CMBN 3}
------------------------------------------------------------------------------

(? chi-t *)  ==>

----------------------------------------------------------------------{CMBN 4}
------------------------------------------------------------------------------

(? f 3 (crpr 1 7 2 (loop3 2 (+ 32 64 128) 2 4 (+ 32 64 128) -2)))  ==>

----------------------------------------------------------------------{CMBN 3}
<-1 * <TnPr 0 1 3 <<Loop[2-1-0 3][1-0 7][2-1 224\2][3-2 224\-2]>>>>
<1 * <TnPr 0 1 3 <<Loop[2-1-0 3][2-0 7][2-1 224\2][3-1 224\-2]>>>>
------------------------------------------------------------------------------

(? chi-t *)  ==>

----------------------------------------------------------------------{CMBN 4}
------------------------------------------------------------------------------

(? f 3 (crpr 4 7 2 (loop3 2 (+ 32 64 128) 2 1 (+ 32 64 128) -2)))  ==>

----------------------------------------------------------------------{CMBN 3}
<1 * <TnPr 0 1 3 <<Loop[2-1-0 3][1-0 7][2-1 224\2][2-0 224\-2]>>>>
<-1 * <TnPr 0 1 3 <<Loop[2-1-0 3][2-0 7][2-1 224\2][1-0 224\-2]>>>>
<1 * <TnPr 2 7 1 <<Loop[0 192\2][224\-2]>>>>
------------------------------------------------------------------------------

(? chi-t *)  ==>

----------------------------------------------------------------------{CMBN 4}
------------------------------------------------------------------------------

(? f 3 (crpr 1 7 2 (loop3 2 (+ 32 64 128) -2 4 (+ 32 64 128) 2)))  ==>

----------------------------------------------------------------------{CMBN 3}
<-1 * <TnPr 0 1 3 <<Loop[2-1-0 3][1-0 7][2-1 224\-2][3-2 224\2]>>>>
<1 * <TnPr 0 1 3 <<Loop[2-1-0 3][2-0 7][2-1 224\-2][3-1 224\2]>>>>
------------------------------------------------------------------------------

(? chi-t *)  ==>

----------------------------------------------------------------------{CMBN 4}
------------------------------------------------------------------------------

(? f 3 (crpr 4 7 2 (loop3 2 (+ 32 64 128) -2 1 (+ 32 64 128) 2)))  ==>

----------------------------------------------------------------------{CMBN 3}
<1 * <TnPr 0 1 3 <<Loop[2-1-0 3][1-0 7][2-1 224\-2][2-0 224\2]>>>>
<-1 * <TnPr 0 1 3 <<Loop[2-1-0 3][2-0 7][2-1 224\-2][1-0 224\2]>>>>
<1 * <TnPr 2 7 1 <<Loop[0 192\-2][224\2]>>>>
------------------------------------------------------------------------------

(? chi-t *)  ==>

----------------------------------------------------------------------{CMBN 4}
------------------------------------------------------------------------------
\end{verbatim}}
We verify now
the required property of the homotopy operator $\chi_{\times\tau}$. As usual, we build
by composition a morphism {\tt zero}.
{\footnotesize\begin{verbatim}
(setf dfr (bndr (twisted-crts-prdc delta)))   ==>

[K24 Morphism (degree -1)]

(setf zero (i-sbtr (idnt-mrph (sorc dfr)) (cmps dfr chi-x)
                   (cmps chi-x dfr)))   ==>

[K33 Morphism (degree 0)]

(setf simplx (crpr 4 7 2 (loop3 2 (+ 32 64 128) -2 1 (+ 32 64 128) 2)))  ==>

<CrPr 2 7 1 <<Loop[1 224\-2][0 224\2]>>>

(? zero 3 simplx)

----------------------------------------------------------------------{CMBN 3}
------------------------------------------------------------------------------
\end{verbatim}}
Then, we do the same test for the homotopy operator $\chi_{\otimes t}$.
{\footnotesize\begin{verbatim}
(setf tw (twisted-tnsr-prdc d))  ==>

[K55 Chain-Complex]

(setf zero-2  (i-sbtr (idnt-mrph tw) (cmps tw chi-t) (cmps chi-t tw)))  ==>

[K64 Morphism (degree 0)]

(setf simpx2 (tnpr 0 1 3 (loop3 0 (mask 5) 2 0 (* 32 (mask 5)) -1)))  ==>

<TnPr 1 <<Loop[31\2][992\-1]>>>

(time (? zero-2 3 simpx2))  ==>

; cpu time (non-gc) 6,050 msec user, 260 msec system
; cpu time (gc)     450 msec user, 0 msec system
; cpu time (total)  6,500 msec user, 260 msec system
; real time  6,903 msec
; space allocation:
;  947,980 cons cells, 0 symbols, 71,344 other bytes

----------------------------------------------------------------------{CMBN 3}
------------------------------------------------------------------------------
\end{verbatim}}
Note nevertheless that the relation is not satisfied for the base generator
$*\otimes *$, namely the tensor product of both base generators.
This is due to the fact that both contractions,  $\chi_{\times t}$ and  $\chi_{\otimes t}$,
correspond to  reductions over  $\mathbb{Z}$ and not on $0$.
{\footnotesize\begin{verbatim}
(setf gnrt-0 (tnpr 0 1 0 +null-loop+))  ==>

<TnPr 1 <<Loop>>>

(? zero-2 0 gnrt-0)  ==>

----------------------------------------------------------------------{CMBN 0}
<1 * <TnPr 1 <<Loop>>>>
------------------------------------------------------------------------------
\end{verbatim}}
Let us build, for instance, the reduction of the twisted tensor product over $\mathbb{Z}$ and check it.
In that case, the two homomorphisms
$f$ and $g$ are respectively the {\em augmentation} and {\em coaugmentation}
morphisms.
$$
\diagram{
{{\cal C}_*(B) \otimes_t {\cal C}_*(GB)} & \stackrel{\chi_{\otimes t}}{\longrightarrow} &
  {}^s{{\cal C}_*(B) \otimes_t {\cal C}_*(GB)} \cr
 {\scriptstyle aug} \downarrow \uparrow {\scriptstyle coaug}  \cr
 {{\cal C}_*(\mathbb{Z})} \cr
}
$$
We recall that, the unit chain complex corresponding to $\mathbb{Z}$ is built by the function
{\tt z-chcm}. The unique generator in degree $0$ is {\tt :z-gnrt}.
{\footnotesize\begin{verbatim}
(setf aug (build-mrph
            :sorc tw
            :trgt (z-chcm)
            :degr 0
            :intr #'(lambda (degr gnrt)
                       (if (zerop degr)
                           (term-cmbn 0 1 :z-gnrt)
                           (zero-cmbn degr)))
            :strt :gnrt
            :orgn '(aug tw)))  ==>

[K67 Cohomology-Class (degree 0)]

(setf coaug (build-mrph
              :sorc (z-chcm)
              :trgt tw
              :degr 0
              :intr #'(lambda (degr gnrt)
                         (if (zerop degr)
                             (term-cmbn 0 1 (tnpr 0 1 0 +null-loop+))
                             (zero-cmbn degr)))
              :strt :gnrt
              :orgn '(coaug tw)))  ==>

[K68 Morphism (degree 0)]

(setf rdct (build-rdct :f aug
                       :g coaug
                       :h Chi-t
                       :orgn '(reduction-tw-Z)))   ==>

[K69 Reduction]

(pre-check-rdct rdct)  ==>

---done---

(setf *tc* (cmbn 3 1 (tnpr 1 3 2 (loop3 0 15 2))))  ==>

----------------------------------------------------------------------{CMBN 3}
<1 * <TnPr 3 <<Loop[15\2]>>>>
------------------------------------------------------------------------------

(setf *bc* (cmbn 0 1 :z-gnrt))  ==>

----------------------------------------------------------------------{CMBN 0}
<1 * Z-GNRT>
------------------------------------------------------------------------------

(check-rdct)  ==>

*TC* =>
----------------------------------------------------------------------{CMBN 3}
<1 * <TnPr 3 <<Loop[15\2]>>>>
------------------------------------------------------------------------------

*BC* =>
----------------------------------------------------------------------{CMBN 0}
<1 * Z-GNRT>
------------------------------------------------------------------------------

Checking *TDD* = 0
Result:
----------------------------------------------------------------------{CMBN 1}
------------------------------------------------------------------------------

              ....... All results null ......

Checking *HG* = 0
Result:
----------------------------------------------------------------------{CMBN 1}
------------------------------------------------------------------------------
\end{verbatim}}

\newpage

\section {An unproved property of the algebraic Kan contraction}

Since the early version of the software (1990), and through a host of examples,
an amazing algebraic property of the Kan contraction $\chi_{\otimes t}$ has
been experimentally discovered. Many computations have been done of
$\chi_{\otimes t}(b \otimes g)$, for various choices of $b \otimes g$
and in all these cases,  as soon as $degree(b) > 0$, then the result is null.
In other words,
$\chi_{\otimes t}$ {\em is null outside the base fiber.}
Up to now, no actual proof of this result has yet  been given. \par
We show that, using the ``soft'' version of $\bar{\Delta}$.

{\footnotesize\begin{verbatim}
(setf db (soft-deltab))  ==>

[K1 Simplicial-Set]

(setf tt (tnpr-contraction db))  ==>

[K54 Morphism (degree 1)]

(setf gnrt1 (tnpr 3 (code '(0 1 2 3 4))
                 4 (loop3 0 (code '(0 1 2 3 4 5)) 2)))  ==>

<TnPr 3 0-1-2-3-4 4 <<Loop[0-1-2-3-4-5\2]>>>

(? tt 7 gnrt1)  ==>

----------------------------------------------------------------------{CMBN 8}
------------------------------------------------------------------------------

(setf gnrt2 (tnpr 4 (code '(2 4 6 8 10 12))
                  5 (loop3 0 (code '(1 3 5 7 9 11 13)) 5)))  ==>

<TnPr 4 2-4-6-8-10-12  5 <<Loop[1-3-5-7-9-11-13\5]>>>

(? tt 9 gnrt2)  ==>

----------------------------------------------------------------------{CMBN 10}
------------------------------------------------------------------------------
\end{verbatim}}
\newpage
But if $degree(b) = 0$, then the result is non--null in general.
{\footnotesize\begin{verbatim}
(? tt 2 (tnpr 0 (code '(0)) 2 (loop3 0 (code '(0 1 2 3)) 1)))  ==>

----------------------------------------------------------------------{CMBN 3}
<-1 * <TnPr 0 <<Loop[2-1-0 0-1][1-0 0-1-2][2 0-1-2-3]>>>>
<1 * <TnPr 0 <<Loop[2-1-0 0-1][2-0 0-1-2][1 0-1-2-3]>>>>
<1 * <TnPr 0-2 <<Loop[0-1-2-3]>>>>
<-1 * <TnPr 0-1-2 <<Loop[1-2-3]>>>>
<-1 * <TnPr 0-1-2-3 <<Loop>>>>
------------------------------------------------------------------------------

(? tt 4 (tnpr 0 (code '(0)) 4 (loop3 0 (code '(1 2 3 4 5 6)) 3)))  ==>

----------------------------------------------------------------------{CMBN 5}
<1 * <TnPr 1 <<Loop[4-3-2-1-0 1-2][3-2-1-0 1-2-3][4-2-1-0 1-3-4][4-3-1-0 1-4-5]
                   [4-3-2 1-4-5-6]>>>>
<1 * <TnPr 1 <<Loop[4-3-2-1-0 1-2][3-2-1-0 1-2-3][4-2-1-0 1-3-4][4-3-1-0 1-4-5]
                   [4-3-2 1-4-5-6\2]>>>>
... ... ... ... ... ... ... ... ... ... ... ... ... ... ... ... ... ... ...
------------------------------------------------------------------------------

(length *)  ==>

537
\end{verbatim}}

\subsection* {Lisp file concerned in this chapter}

{\tt ls-twisted-products.lisp}.
